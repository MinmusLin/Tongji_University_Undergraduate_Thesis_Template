\section{引用}\label{sec:reference}

\subsection{参考文献}

可使用 \verb|\cite| 命令引用参考文献。\cite{MSSB201404007}

如果需要同时标注多个参考文献,可使用逗号分隔键值。\cite{MSSB201404007,TJLT200705022}

使用 \verb|\parencite{key1,key2,key3...}| 命令可以在正文中产生带有括号的引用参考文献。如\parencite{MSSB201404007}。

可使用 \verb|\nocite{key1,key2,key3...}| 将参考文献条目加入文献表中,但不在正文中引用。使用 \verb|\nocite{*}| 可将参考文献数据库中的所有条目加入文献表中。

当我们想在正文(非参考文献章节)中插入对某一参考文献的完整引用时,可以使用 \verb|\fullcite{key1}| 命令。下面是使用 \verb|\fullcite| 命令的引用示例:

\fullcite{MSSB201404007}

有时,我们想要用脚注的形式引用某参考文献。\footfullcite{MSSB201404007}

\subsection{脚注}

脚注是一种在文本底部添加注释或补充说明的方式\footnote{通常,我们在脚注里也写完整的句子。在文本中使用脚注时,应该遵循学术规范,尽可能引用可信的来源,并注明出处。}。

\subsection{交叉引用}

请参见第\ref{sec:introduction}章节。

请参见第\pageref{sec:introduction}页。

请参见\cref{sec:intro}。

请参见\cref{sec:intro,sec:math}。

如果需要修改引用词汇,可以在导言区添加如下代码:

\begin{Verbatim}
    \crefname{对象类型}{引用词汇}{引用词汇复数形式}
    \Crefname{对象类型}{引用词汇}{引用词汇复数形式}
\end{Verbatim}

例如,可以使用以下代码将“定理”引用词汇修改为“命题”:

\begin{Verbatim}
    \crefname{theorem}{命题}{命题}
    \Crefname{theorem}{命题}{命题}
\end{Verbatim}